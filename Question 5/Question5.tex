\documentclass[addpoints]{exam}
\usepackage{amsmath}
\usepackage{amssymb}
\usepackage{listings}
\usepackage{pxfonts}
\usepackage{xcolor}


\pagestyle{headandfoot}
\firstpageheadrule
\runningheadrule
\firstpageheader{Coursework}{}{Sam Robbins}
\runningheader{Coursework}{Algorithms and Data Structures}{Sam Robbins}
\firstpagefooter{}{}{}
\runningfooter{}{}{}
\renewcommand{\solutiontitle}{\noindent\textbf{My Solution:}\par\noindent}


\printanswers
\usepackage{graphicx}
\marksnotpoints
\bracketedpoints
\pointsdroppedatright
\pointsinrightmargin
\begin{document}
	\begin{center}
		\underline{\LARGE Algorithms and Data Structures Coursework}
	\end{center}
	\begin{questions}
		\setcounter{question}{4}
		\question For each of the following recurrences, give an expression for the runtime $T(n)$ if the recurrence can be solved with the Master Theorem. Otherwise, state why the Master theorem cannot be applied. You should justify your answers.
		\begin{parts}
			\part[3] $T(n)=9T(n/3)+n^2$
			\begin{solution}[2in]
	$$T(n)=9T\bigg(\dfrac{n}{3}\bigg)+n^2$$
$a=9, b=3, f(n)=n^2, \log_ba=2$
$$f(n)=\Theta(n^2)$$
$$T(n)=\Theta(n^2\log n)$$
			\end{solution}	
		
		
			\part[3] $T(n)=4T(n/2)+100n$
			\begin{solution}[2in]
	$$T(n)=4T\bigg(\dfrac{n}{2}\bigg)+100n$$
$a=4,b=2,f(n)=100n, \log_ba=2$
$$f(n)=\mathcal{O}(n^{2-1})$$
$$T(n)=\Theta(n^2)$$
			\end{solution}
		

			\part[3] $T(n)=2^nT(n/2)+n^3$
\begin{solution}[2in]
As a is not a number this cannot be solved using master theorem
\end{solution}


			\part[3] $T(n)=3T(n/3)+c\cdot n$
\begin{solution}[2in]
	Under the assumption that c is a constant, otherwise this cannot be solved using master theorem
$$T(n)=3T\bigg(\dfrac{n}{3}\bigg)+c\cdot n$$
$a=3,b=3,f(n)=c\cdot n,\log_ba=1$	
$$f(n)=\Theta(n^1)$$
$$T(n)=\Theta(n\log n)$$
\end{solution}


			\part[3] $T(n)=0.99T(n/7)+1/(n^2)$
\begin{solution}[2in]
	$$T(n)=0.99T\bigg(\dfrac{n}{7}\bigg)+\dfrac{1}{n^2}$$
$a<1$ so Master theorem cannot be performed.
\end{solution}

		\end{parts}	


	\end{questions}
	
	
	
	
	
\end{document}




